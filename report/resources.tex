% !TEX root = proposal.tex


\section{Resources}

\subsection{Organizing team}


%% \templatedoc{
%% Provide a short biography of all team members, stressing their competence for their assignments in the competition organization. Please note that diversity in the organizing team is encouraged, please elaborate on this aspect as well.
%% }


The organization is undertaken by a diverse team from across three continents, mixing partners from three universities and four companies. 


%% \matthijs{Maybe better to organize by person and specify who will be involved in what tasks}

\paragraph{Martin Aum\"uller} is an assistant professor at IT University of Copenhagen,
working on the design and analysis of randomized algorithms for similarity search problems.
He is also a main contributor to the benchmarking framework \emph{http://ann-benchmarks.com}. 
He will be involved as \emph{beta tester} and in the \emph{evaluation framework setup}.

\paragraph{Matthijs Douze} 
is a research scientist at Facebook AI research, with about 10 years of experience in the domain of ANNS. 
He will be involved as a \emph{data provider} and a \emph{baseline method provider}.

\paragraph{Artem Babenko} 
is a research scientist at Yandex, Russia. He has developed several ANNS algorithms, most of which address billion-scale problems on a single machine.
He will be involved as a \emph{data provider}.

\paragraph{Dmitry Baranchuk}
is a researcher at Yandex who has worked on graph-based methods for ANNS and developed one of the most recent in-memory indices for billion-scale search. He will be involved as a \emph{beta tester} and \emph{evaluator}.

\paragraph{Ravishankar Krishnaswamy} is a researcher at Microsoft Research India and adjunct faculty member at IIT Madras. His research interests are broadly in algorithms and optimization, and recently he has been involved in designing scalable ANNS algorithms for real-world scenarios. He will be assisting the organization as \emph{evaluator}, \emph{beta tester} for  the competition platform, and \emph{baseline provider}.

\paragraph{Harsha Simhadri} 
is a researcher at Microsoft Research India who works on developing
practical algorithms. He has developed new ANNS algorithms for
out-of-core serving that are used in production.  He will be involved
as a \emph{coordinator}, \emph{evaluator}, \emph{platform
  administrator}, \emph{beta tester}, and \emph{baseline method
  provider}.
  
 \paragraph{Gopal Srinivasa} is a Research SDE at Microsoft Research India. He has worked on deploying ANNS to many industry scenarios and will assist the organizers as an \emph{evaluator}.

\paragraph{George Williams} is a systems engineering and data science lead at GSI Technology. He designs and develops applications for custom in-memory associative micro-processors. He will be helping out with the bare-metal infrastructure and systems management and will serve as a \emph{platform administrator} and \emph{evaluator}.

\paragraph{Jingdong Wang}
is a researcher at Microsoft Research Asia, China. He has been working on graph-based methods and quantization-based methods for ANNS. His algorithm NGS (ACM MM 2012) was widely used in Bing and was later improved and open-sourced as SPTAG~\url{https://github.com/microsoft/SPTAG}. 
He will be involved as an~\emph{evaluator} and a~\emph{beta tester}.

\paragraph{Qi Chen} is a researcher at Microsoft Research Asia. Her research interests are in cloud computing, distributed systems, and algorithms. In the recent three years, cooperating with Jingdong Wang, she has developed scalable ANN algorithms for super-large scale real-world vector search scenarios. She will be involved as a~\emph{data provider} and an~\emph{evaluator}. 

\paragraph{Suhas Jayaram Subramanya} is a Ph.D student at Carnegie Mellon University. His research interests lie in the broad intersection of machine learning and systems, with recent contributions in developing SSD-based ANNS algorithms. He will be assisting the organizers as a \emph{beta tester}, \emph{baseline provider} and \emph{evaluator}.

\iffalse

%% \begin{itemize}
%% \item coordinators
%% \item data providers
%% \item platform administrators
%% \item baseline method providers
%% \item beta testers
%% \item evaluators
%% \end{itemize}

\fi


\subsection{Resources provided by organizers, including prizes}

%% Describe your resources (computers, support staff, equipment, sponsors, and available prizes and travel awards). 

\noindent {\bf Compute}. Microsoft Research has committed to providing
\$50,000 in Azure Compute credit for teams to develop code.  

\noindent {\bf Bare-Metal Compute}.  Microsoft Research will also
provide the bare-metal machines on which the final evaluation is
performed for T2,T3. GSI Technology has also committed to providing
access to bare-metal servers and datacenter infrastructure as well as
system and IT resources, as needed for the T3 competition.

%% For live/demonstration competitions, explain how much will be provided by the organizers (demo framework, software, hardware) and what the participants will need to contribute (laptop, phone, other hardware or software).

\subsection{Support requested}


We request NeurIPS for a 5 hour session(s) for (a) discussing the
results of the competition, (b) the teams with the best entries to
present their solutions to other participants and NeurIPS audience at
large, and (c) invited talks.

%% \templatedoc{
%% Please indicate the kind of support you need from the conference and \textbf{keep in mind that NeurIPS 2021 is a \emph{virtual-only} conference.}
%% }

% For live/demonstration competitions, indicate what you will need in order to run the live competition. We do not commit to provide all such support free-of-charge.

