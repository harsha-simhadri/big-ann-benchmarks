
\section{Introduction}
\label{sec:intro}

%\matthijs{Formatting instructions here: \url{}}

Approximate Nearest Neighbor (ANN) search is an important tool in various fields including computer vision, 
natural language processing, information retrieval, and retrieval-augmentation.
For example, in the context of Large-Language-Models (LLMs), ANN search is used to add knowledge after model training~\cite{DBLP:conf/iclr/KhandelwalLJZL20} via retrieval-augmented generation.
The necessary similarity search operations such as \emph{nearest neighbor queries}
are often required on large datasets, often with  \emph{billions} of \emph{high-dimensional, real-valued} vectors,
and response times in milliseconds are needed for LLMs to use these in multi-turn reasoning.
As result, efficient and accurate ANN search algorithms are essential.

As ANN search becomes commonplace, many variants have become critical  in practice.
For example, database
queries use a combination of vector similarity and predicates over attributes. Multi-modal
search involves vectors representing different modalities and thus potentially different distributions.
New sparse embedding models are being invented for interpretability and to incorporate text search~\cite{formal2022splade}.
Indices are continually updated to reflect changing content and database transactions.
These complex scenarios are the current reality in the industry, and require
indices that work well in constrained  computational environments.


% The 2023 Big ANN Challenge, 
% hosted at 
% aimed to push the boundaries of current indexing and search methodologies
% by addressing four challenging variants of ANN search: filtered search, out-of-distribution data, sparse vectors, and streaming scenarios.
% These variants represent realistic and complex scenarios encountered in practical applications,
% moving beyond the well-trodden path of standard dense vector indexing.

Our goal was to shine more light on these variants through a competition with new datasets and baselines,
and encourage the research community to develop  new indexing and search algorithms, and their optimized implementations.
To ensure broad participation and accessibility, the scale of the tasks in the competition was chosen
to be large enough to be interesting and small enough to experiment on laptops, small workstations, or virtual machines.
The datasets were carefully curated to be representative yet manageable in size, and the evaluation was
conducted on standardized Azure virtual machines with limited computational power and memory.
Small grants for cloud compute credits provided by Pinecone and AWS further encouraged participation.
The competition emphasized open-source contributions, promoting transparency and reproducibility.

This paper summarizes the competition, detailing the specific tracks and datasets used (Section~\ref{sec:tracks}),
the evaluation metrics employed (Section~\ref{sec:eval}), and the notable approaches taken by the participants (Section~\ref{sec:results}). 
By highlighting the advancements made during the challenge, we aim to provide valuable insights 
into the current state of ANN research and identify promising directions for future work.

%By addressing these specific challenges, the competition aimed to stimulate innovative solutions and attract participation from both academic and industrial communities.

\paragraph{Broader Impact.}  
While the previous NeurIPS’21 competition on billion-scale approximate nearest neighbor search \cite{DBLP:conf/nips/SimhadriWADBBCH21} focused on establishing datasets and the experimental methodology for evaluating large-scale ANN search systems, the present paper proposes novel, industry-motivated search tasks and evaluates the state of the art. We establish clear task definitions, suggest datasets and workloads for them, and introduce the experimental framework that defines the methodology. 
We believe that this competition had positive impact on this research community.
By using small datasets and accessible hardware, as well as issuing generous grants for development, the competition ensured that anyone could participate regardless of their own resources.
After the competition, people used our proposal in their own research, see for example \cite{DBLP:conf/sigir/BruchNRV24,DBLP:journals/debu/KrishnaswamyMS24,DBLP:journals/corr/abs-2502-13826,DBLP:journals/corr/abs-2502-13245,upreti2025costeffectivelowlatencyvector}.
Moreover, the detailed description of competition entries led to top-tier publications such as~\cite{DBLP:journals/pvldb/ChenZHJW24}.

\paragraph{Limitations.}
Applications of ANN search, such as ranking or recommendation, can be used towards unethical ends.
However, this competition focuses on developing faster algorithms for existing problems,
and does not meaningfully enhance any existing capacity for unethical behavior. The limitations of this work are inherent to the task of creating a competition with well-defined evaluation metrics: 
the metrics and tracks cannot capture every nuance of a robust ANN search algorithm. 
However, the tracks captured diverse scenarios and used the most widely accepted evaluation
metrics in the community.
%\magdalen{I added this paragraph to comply with the checklist.}